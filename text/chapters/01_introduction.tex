% !TEX root ../thesis
\chapter{Introduction\label{ch:Introduction}}

\abstract{In this chapter we describe the motivation and aims of the thesis as well as preview our contributions.
First, we explain why \glspl{bn} pose a threat to the Internet and motivate why \emph{Dridex} should be the focus of research.
Then we briefly summarize the contributions of this thesis and give a general outline of the remaining chapters.}

\section{Motivation\label{sec:Introduction::Motivation}}
\Glspl{bn} are classified as an ever-growing risk to \acrshort{it} security.
A cluster of compromised hosts (\glspl{bot}) is controlled by a \gls{c2} operated by the so-called \gls{bm}~\cite{aburajab2006multifaceted}.
Malicious software (\gls{mw}) is installed on the machines which listens for commands from the \gls{c2} and executes them in the host context.
These \glspl{bot} are then used to perform various malicious actions ranging from spam email delivery and distributed password cracking to full-blown \gls{ddos} attacks using the combined bandwidth available to all available \glspl{bot}.

When the \gls{mw} collects credentials or credit card information from the compromised host it is usually sent back to the \gls{c2}.
While traditional variants send the data directly to the server, \gls{p2p} \glspl{bn} are often organized hierarchically and relay the data to the \gls{c2}.
This helps the \gls{bm} to stay hidden behind multiple layers of interconnected \glspl{bot} and prevent detection by law enforcement.

It is therefore very important to gain insights into the communication protocols used in these sophisticated \glspl{bn} as well as their capabilities in regards to computing power and bandwidth.
This knowledge can then be used to find potential design or implementation flaws in these hugely distributed systems which might aid in global takedown attempts.
As the source code for \glspl{bn} is not freely available and communication with the \gls{c2} is heavily encrypted, the only way to gain insights into the protocols is through reverse engineering the malicious binary samples.
However, \gls{mw} authors often implement a number of techniques to disrupt binary analysis such as obfuscation and packing.
We estimate a significant time will be spent circumventing these protective measures until the relevant parts for message handling can be freely analyzed.

In this thesis, we aim to reverse engineer and analyze one of the most dangerous \gls{p2p} \glspl{bn} today, the Dridex banking trojan.
With over £20 million stolen in the United Kingdom alone (as of 2015), it is one of the most financially damaging trojans~\cite{nca2015uk}.
Often cited for it
Dridex is constantly adapted and updated to remain undetected and the \glspl{bm} behind it are often exploring new alternative revenue streams.
While it started with classical credential stealing, a recent \gls{mw} strain was used to deliver ransomware.
This broadens the target audience from online banking users to potentially every computer user worldwide.


\section{Contributions\label{sec:Introduction::Contributions}}
The reverse engineering process will be largely aided by IDA Pro\fnote{{\url{https://www.hex-rays.com/products/ida/overview.shtml}}}, a professional disassembling tool with advanced reconstruction abilities, as well as other standard tools for debugging, disassembly and other low-level tasks.
This will be essential in understanding the obfuscated binary code of Dridex.
The focus is hereby placed on the communication protocols as they represent the entry point in monitoring the \gls{bn}.
The goal is to provide byte-level descriptions of as many messages as possible which can be used to directly construct valid payloads.
Although the formats of those messages are updated (and hardened) from time to time, the general structure and included information should be useful in bootstrapping further research in the future.
To validate the results obtained from reverse engineering a scanner will be implemented capable of detecting Dridex \glspl{sp} in large \gls{ip}-address ranges.
Finally, the \emph{Dridex L2 Scanner} will then be invoked on some representative subnets to gain some knowledge about the \gls{bn}'s population count.


\section{Outline\label{sec:Introduction::Outline}}
In this first chapter, we highlighted the danger of \glspl{bn}, motivated analysis, takedowns and the Dridex L2 Scanner.
The second chapter summarizes previous research and publications on the topic of \glspl{bn} in general and Dridex in particular.
In \autoref{ch:Reverse_engineering_Dridex} we present the results of the reverse engineering efforts with a special focus on the communication protocols.
In the following chapter, we briefly discuss the scanner's implementation, present our data set of \gls{ip}-address ranges and evaluate the results obtained by the scan.
The final chapter summarizes the work of this thesis and proposes future research based off it.