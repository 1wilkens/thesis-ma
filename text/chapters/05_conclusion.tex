% !TEX root ../thesis
\chapter{Conclusion\label{ch:Conclusion}}

This thesis analyzed the Dridex \gls{bn} and presented descriptions of its architecture and communication protocols.
With the knowledge acquired we were able to detect and monitor a limited set of \glspl{sp}/layer 2 (L2) nodes which indicates that these host actually communicate through the protocols discovered.


\section{Results}\label{sec:Conclusion::Results}
From an architectural standpoint we covered the whole infection process starting from the dropper up to the complete \gls{bot} stage.
Every component was described as accurately as possible with an extended focus on the main bot module as it is responsible for the communication with the \gls{bn} and \gls{c2}.
We briefly documented the several functionalities provided by the module as well as a selection of obfuscation techniques that were encountered.
The focus was placed on byte-exact representations of all network communication to bootstrap further research.

To efficiently detect L2 nodes in large \gls{ip}-address ranges we chose to emulate the \emph{CheckMe} process.
This part of the \gls{p2p} protocol forces the \gls{sp} to attempt a connection to the scanning host.
Correlating the \gls{ip} addresses that returned a proper, protocol conforming message with the hosts that tried to establish a connection during the scan provides a strong \gls{ioc}.
For the monitoring component the `\lstinline|ping|' request was chosen as it increases response time allowing for quicker uptime probing.

As Great Britain was a major target of past Dridex campaigns, we performed a scan of all large subnets allocated to this region.
Although the analyzed sample, and therefore the campaign, was released 8 month ago, we were able to verify 42 total \glspl{sp} across Europe.
We monitored the detected L2 nodes, building up an uptime profile throughout the day and retrieving their unique \emph{BotId}.
The results obtained supported the theory that a subset of \gls{sp} are running on office machines which are turned off in the evening.
Finally, a potential vulnerability and an information leak were discovered which might aid in future takedown attempts.


\section{Future work}\label{sec:Conclusion::Future_work}
While this thesis provides a solid basis for understanding and monitoring the malware, several aspects should be covered in more detail in future work.
Based off the knowledge presented, certain questions remain open.

\begin{description}
    \item[Continuous monitoring]
    Although the scanning and crawling were highly important to verify the basic understanding of the \gls{p2p} communication, the gathered data remains inconclusive for real assumptions about Dridex's size and reach.
    To gain further insights into the \gls{bn}'s population counts, longer, continuous monitoring is essential.
    This would also help to detect changes and updates in the protocol as it has already evolved several times in the past.
    \item[Binary and \gls{pl} updates]
    The mechanisms and protocol messages for both binary and \gls{pl} updates were out of scope for this thesis and propose a promising research question.
    As these processes change the state on the infected host, they could contain a vulnerability to potentially take down Dridex.
    Additionally these requests are essential for a traditional crawling approach to deal with \gls{node_churn} without having to rescan the whole Internet periodically.
    \item[Prevention]
    Details about the loader stage and especially the memory injection technique and target processes were not fully analyzed in this thesis as they do not directly influence communication protocols.
    This limits the results' usefulness in prevention scenarios particularly on end-user machines which (usually) cannot be scanned from the open Internet.
    Future research could help identify Dridex during the infection process for example through registry keys or the initial \gls{pl} download from the distribution server.
    \item[Vulnerabilty analysis]
    While we discovered a weakness in length validations in several functions handling the \gls{p2p} protocol, it remains unclear, if these can be exploited in a meaningful way to manipulate or cleanup a Dridex infection.
    In combination with the research about the update process mentioned above, this could assist in a future takedown attempt.
\end{description}