\chapter*{Abstract}\label{ch:Abstract}

\thispagestyle{empty}

In the last years \glspl{bn} present a growing threat to end-users and network administrators alike.
The large networks of infected hosts are able to deliver massive \gls{ddos} attacks and allow the \gls{bm} to exfiltrate data from the \glspl{bot} in nearly untraceable ways.
With the evolution from centralized architectures to \gls{p2p} networks \gls{bn} became more resilient to takedown attempts by law enforcement.
As monitoring is more important then ever to evaluate the threats posed by them.
Because of their resilience, exploiting vulnerabilities in communication protocols is often the only way to take down a \gls{bn}.

This thesis analyzes and monitors one of the most dangerous \glspl{bn} today, the Dridex financial trojan, with a focus on its \gls{p2p} protocol.
It provides byte-level descriptions of request and response messages to bootstrap further research.
Addtitionally, details of the sophisticated module execution process as well as the bot main module's internal architecture are presented.
To verify our findings about the \gls{mw} the \emph{Dridex L2 Scanner} was developed, applied to large \gls{ip}-address ranges from Great Britain and Europe and revealed 42 total \glspl{sp} which were then monitored for a short timespan.\\
%
From a protocol standpoint no immediate vulnerabilities could be found in the analyzed messages of Dridex's \gls{p2p} communication besides a minor information leak.
Future research should continue this research to support future takedown attempts.
